\input{paquetes.tex}

\title{	
\normalfont \normalsize 
\textsc{\textbf{Modelos de Computación} \\ Grado en Ingeniería Informática \\ Universidad de Granada} \\ [25pt] % Your university, school and/or department name(s)
\horrule{0.5pt} \\[0.4cm] % Thin top horizontal rule
\huge Memoria Prácticas \\ % The assignment title
\horrule{2pt} \\[0.5cm] % Thick bottom horizontal rule
}

% Nombre y apellidos
\author{ 
    Marina Muñoz Cano
    \\
    Mario López González
} 

\date{\normalsize\today} % Incluye la fecha actual

%----------------------------------------------------------------------------------------
% DOCUMENTO
%----------------------------------------------------------------------------------------

\begin{document}

\maketitle % Muestra el Título

\newpage %inserta un salto de página

\tableofcontents % para generar el índice de contenidos
\listoffigures % para generar el índice de imágenes

\newpage

\section{Practica 1 - Relación Ejercicios 1b}

\subsection{Ejercicios Sencillos}

\textbf{a)}  $\{ u \in \{0,1\}^{\ast} $ tales que $\mid u \mid$ $\leq 4 \}$

\begin{figure}[H] 
	\centering
	\includegraphics[scale=0.35]{../practica_1/images/sencillo_a.png} 
	\caption{Sencillo a en JFLAP} 
    \label{fig:sencillo_a}
\end{figure}

Usamos una variable X que puede tomar valor 0, 1 o $\varepsilon$, como son palabras de 0's y 1's de longitud menor o igual que 4, 
el simbolo de partida son cuatro X's. De esta forma se aceptan cadenas de longitud 4 o menos, ya que X puede tomar el valor $\varepsilon$. \\

\textbf{b)}  Palabras con 0's y 1's que no contengan dos 1's consecutivos y que empiecen por un 1 y que terminen por dos 0's.

\begin{figure}[H] 
	\centering
	\includegraphics[scale=0.35]{../practica_1/images/sencillo_b.png} 
	\caption{Sencillo b en JFLAP} 
	
    \label{fig:sencillo_b}
\end{figure}

La variable A añade 0's tras añadir un 1, para que no contenga dos consecutivos. La variable B va generando el resto de la cadena (Añadiendo 0's o 1)
de forma que la unica forma de acabar la cadena es cuando B vale 0, es decir, con dos 0 consecutivos. 
\\

\textbf{c)}  El conjunto vacío. 

\begin{figure}[H] 
	\centering
	\includegraphics[scale=0.5]{../practica_1/images/sencillo_c.png} 
	\caption{Sencillo c en JFLAP} 
	
    \label{fig:sencillo_c}
\end{figure}

No mostramos ejemplos, ya que JFLAP no nos lo permite, al no detectar ningun simbolo como terminal.

\begin{figure}[H] 
	\centering
	\includegraphics[scale=0.5]{../practica_1/images/sencillo_c_err.png} 
	\caption{Sencillo c error en JFLAP} 
	
    \label{fig:sencillo_c_err}
\end{figure}

\textbf{d)}  El lenguaje formado por los números naturales.

\textbf{f)}  $\{ a^{n} b^{2n} c^{m} \in \{a,b,c\}^{\ast}$ con $n,m > 0\}$

\textbf{g)}  $\{ a^{n} b^{m} c^{n} \in \{a,b\}^{\ast}$ con $m,n > 0\}$

\textbf{h)}  Palabras con 0's y 1's que contengan la subcadena 00 y 11.

\textbf{i)}  Palíndromos formados con las letras a y b.

\subsection{Ejercicios Intermedios}

\textbf{a)}  $\{ uv \in \{0,1\}^{\ast} $ tales que $u^{-1}$ es un prefijo de $v\}$

\textbf{b)}  $\{ ucv \in \{a,b,c\}^{\ast} $ tales que $u$ y $v$ tienen la misma longitud$\}$

\textbf{c)}  $\{ u1^{n} \in \{0,1\}^{\ast} $ donde $\mid u \mid$ $= n \}$

\textbf{d)}  $\{ a^{n} b^{n} a^{n+1} \in \{a,b\}^{\ast}$ con $n > 0\}$

\subsection{Ejercicios Dificiles}

\textbf{a)}  $\{ u0v \in \{0,1\}^{\ast} $ tales que $u^{-1}$ es un prefijo de $v\}$

\subsection{Ejercicios Extremos}

\textbf{a)}  $\{ ww$ con $w \in \{0,1\}^{\ast}\}$

\section{Practica 2 - Analizador Léxico}

\section{Practica 3 - Codificador-Decodificador}


\end{document}
