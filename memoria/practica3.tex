\input{paquetes.tex}

\title{	
\normalfont \normalsize 
\textsc{\textbf{Modelos de Computación} \\ Grado en Ingeniería Informática \\ Universidad de Granada} \\ [25pt] % Your university, school and/or department name(s)
\horrule{0.5pt} \\[0.4cm] % Thin top horizontal rule
\huge Memoria Prácticas \\ % The assignment title
\horrule{2pt} \\[0.5cm] % Thick bottom horizontal rule
}

% Nombre y apellidos
\author{ 
    Marina Muñoz Cano
    \\
    Mario López González
} 

\date{\normalsize\today} % Incluye la fecha actual

%----------------------------------------------------------------------------------------
% DOCUMENTO
%----------------------------------------------------------------------------------------

\begin{document}

\maketitle % Muestra el Título

\begin{figure}[H] 
	\centering
	\includegraphics[scale=0.045]{chatbot.jpg} 
    \label{fig:chatbot}
\end{figure}

\newpage %inserta un salto de página

\tableofcontents % para generar el índice de contenidos
\listoffigures % para generar el índice de imágenes

\newpage

\section{Practica 1 - Relación Ejercicios 1b}

\subsection{Ejercicios Sencillos}

\textbf{a)}  $\{ u \in \{0,1\}^{\ast} $ tales que $\mid u \mid$ $\leq 4 \}$

\begin{figure}[H] 
	\centering
	\includegraphics[scale=0.35]{../practica_1/images/sencillo_a.png} 
	\caption{Sencillo a en JFLAP} 
    \label{fig:sencillo_a}
\end{figure}

Usamos una variable X que puede tomar valor 0, 1 o $\varepsilon$, como son palabras de 0's y 1's de longitud menor o igual que 4, 
el simbolo de partida son cuatro X's. De esta forma se aceptan cadenas de longitud 4 o menos, ya que X puede tomar el valor $\varepsilon$. \\

\textbf{b)}  Palabras con 0's y 1's que no contengan dos 1's consecutivos y que empiecen por un 1 y que terminen por dos 0's.

\begin{figure}[H] 
	\centering
	\includegraphics[scale=0.35]{../practica_1/images/sencillo_b.png} 
	\caption{Sencillo b en JFLAP} 
    \label{fig:sencillo_b}
\end{figure}

La variable A añade 0's tras añadir un 1, para que no contenga dos consecutivos. La variable B va generando el resto de la cadena (Añadiendo 0's o 1)
de forma que la única forma de acabar la cadena es cuando B vale 0, es decir, con dos 0 consecutivos. 
\\

\textbf{c)}  El conjunto vacío. 

\begin{figure}[H] 
	\centering
	\includegraphics[scale=0.5]{../practica_1/images/sencillo_c.png} 
	\caption{Sencillo c en JFLAP} 
    \label{fig:sencillo_c}
\end{figure}

No mostramos ejemplos, ya que JFLAP no nos lo permite, al no detectar ningún símbolo como terminal.

\begin{figure}[H] 
	\centering
	\includegraphics[scale=0.5]{../practica_1/images/sencillo_c_err.png} 
	\caption{Sencillo c error en JFLAP} 
    \label{fig:sencillo_c_err}
\end{figure}

\textbf{d)}  El lenguaje formado por los números naturales.

El lenguaje genera las cifras del numero final de izquierda a derecha, no puediendo empezar por 0. De esta forma generamos un número que comprende
entre el 1 y el 9. Tras esto se le puede agregar cualquier número entre 0 y 9.

\begin{figure}[H] 
	\centering
	\includegraphics[scale=0.35]{../practica_1/images/sencillo_d.png} 
	\caption{Sencillo d en JFLAP} 
    \label{fig:sencillo_d}
\end{figure}

\textbf{f)}  $\{ a^{n} b^{2n} c^{m} \in \{a,b,c\}^{\ast}$ con $n,m > 0\}$

Con la variable A, por cada a generamos dos b's. Tras generar las n a's y las 2n b's generamos m c's con la variable C.

\begin{figure}[H] 
	\centering
	\includegraphics[scale=0.35]{../practica_1/images/sencillo_f.png} 
	\caption{Sencillo f en JFLAP} 
    \label{fig:sencillo_f}
\end{figure}

\textbf{g)} $\{ a^{n} b^{m} a^{n} \in \{a,b\}^{\ast}$ con $m,n > 0\}$

Con el símbolo de partida S, generamos el número de a's mínimo que admite el lenguaje. Tras esto se puede seguir añadiendo a's a ambos lados o incluir las m b's,
siendo obligatorio añadir como mínimo una b. 

\begin{figure}[H] 
	\centering
	\includegraphics[scale=0.4]{../practica_1/images/sencillo_g.png} 
	\caption{Sencillo g en JFLAP} 
    \label{fig:sencillo_g}
\end{figure}

\textbf{h)}  Palabras con 0's y 1's que contengan la subcadena 00 y 11.

\begin{figure}[H] 
	\centering
	\includegraphics[scale=0.35]{../practica_1/images/sencillo_h.png} 
	\caption{Sencillo h en JFLAP} 
    \label{fig:sencillo_h}
\end{figure}

Con las variables S generamos las dos subcadenas necesarias, representadas por X e Y. Con las variables anteriores se añaden variables Z a ambos lados de las
cadenas, siendo posible sustituilas por 0's, 1's o $\varepsilon$. De esta forma, se generan cadenas de 0's y 1's que contienen en cualquier orden las subcadenas 00 y 11. \\

\textbf{i)}  Palíndromos formados con las letras a y b.

Para formar los palíndromos con el símbolo de partida generamos dos a's o dos b's. En las cadenas generadas incluimos la variable S para que se puedan generar las cadenas de 
a's y b's necesarias para crear el palíndromo. Además, incluimos las reglas de producción que sustituyen S por a y b para poder generar palíndromos impares y otra regla,
S $\rightarrow$ $\varepsilon$, para generar pares.

\begin{figure}[H] 
	\centering
	\includegraphics[scale=0.375]{../practica_1/images/sencillo_i.png} 
	\caption{Sencillo i en JFLAP} 
    \label{fig:sencillo_i}
\end{figure}

\newpage

\subsection{Ejercicios Intermedios}

\textbf{a)}  $\{ uv \in \{0,1\}^{\ast} $ tales que $u^{-1}$ es un prefijo de $v\}$

Partimos de las variables XY, con X generaremos $u$ y $u^{-1}$, y con Y generaremos el resto de v. El palíndromo par, lo generamos de forma
similar al que generamos en el ejercicio sencillo i con la excepción de no formar palíndromos impares.
v se genera a partir de la variable Y con 0's y con 1's sin importar el orden.

\begin{figure}[H] 
	\centering
	\includegraphics[scale=0.325]{../practica_1/images/intermedio_a.png} 
	\caption{Intermedio a en JFLAP} 
    \label{fig:intermedio_a}
\end{figure}

\textbf{b)}  $\{ ucv \in \{a,b,c\}^{\ast} $ tales que $u$ y $v$ tienen la misma longitud$\}$

\begin{figure}[H] 
	\centering
	\includegraphics[scale=0.35]{../practica_1/images/intermedio_b.png} 
	\caption{Intermedio b en JFLAP} 
    \label{fig:intermedio_b}
\end{figure}

Con la producción S $\rightarrow$ ASA vamos generando $u$ y $v$  intercambiando A por a, b, o c y S por ASA de nuevo para incrementar la longitud de $u$ y $v$ . Por último,
intercambiamos S por $c$ cuando $u$ y $v$  son de la longitud deseada. \\

\textbf{c)}  $\{ u1^{n} \in \{0,1\}^{\ast} $ donde $\mid u \mid$ $= n \}$

Con la variable S formamos las reglas de producción más básicas del lenguaje, S $\rightarrow$ 1S1 y S $\rightarrow$ 0S1, con esto conseguimos generar la cadena $u$ deseada 
junto con la cadena de 1's del mismo tamaño que $u$. Para terminar la cadena aplicamos la regla de producción S $\rightarrow$ $\varepsilon$.
 
\begin{figure}[H] 
	\centering
	\includegraphics[scale=0.35]{../practica_1/images/intermedio_c.png} 
	\caption{Intermedio c en JFLAP} 
    \label{fig:intermedio_c}
\end{figure}

\textbf{d)}  $\{ a^{n} b^{n} a^{n+1} \in \{a,b\}^{\ast}$ con $n > 0\}$

\begin{figure}[H] 
	\centering
	\includegraphics[scale=0.35]{../practica_1/images/intermedio_d.png} 
	\caption{Intermedio d en JFLAP} 
    \label{fig:intermedio_d}
\end{figure}

La variable X se mueve hacia la derecha de la ultima b para generar una a y una b, una vez hecho esto, se intercambia por una Y, que se desplaza hasta la izquierda de la primera
b para generar una a y se intercambia por una X. Esto se repite hasta que generamos la cadena al completo.

\subsection{Ejercicios Difíciles}

\textbf{a)}  $\{ u0v \in \{0,1\}^{\ast} $ tales que $u^{-1}$ es un prefijo de $v\}$

Partimos de dos variables X e Y. Con la variable X, vamos generando $u$ y $u^{-1}$, una vez generadas, entre $u$ y $u^{-1}$ añadimos un 0. Con la variable Y generamos el resto de $v$.


\begin{figure}[H] 
	\centering
	\includegraphics[scale=0.35]{../practica_1/images/dificil_a.png} 
	\caption{Difícil a en JFLAP} 
    \label{fig:dificil_a}
\end{figure}

\subsection{Ejercicios Extremos}

\textbf{a)}  $\{ ww$ con $w \in \{0,1\}^{\ast}\}$

Usamos tres variables:
\begin{itemize}
\item \textbf{M}: para ir insertando en orden la cadena w. A la izquierda de M va quedando en orden, mientras que, a su derecha va quedando invertida.
\item \textbf{I}: para ir invirtiendo la parte la izquierda de M. Se va desplazando hacia la izquierda intercambiando dos símbolos terminales entre si.
\item \textbf{F}: marca el final de la cadena de la izquierda para saber cuando I tiene que parar de invertir. 
\end{itemize}
De esta forma, cada vez que se inserta un 1 o 0 con M, la función de I es ir desplazándolo dejándolo al comienzo de la cadena (marcado por F).

\begin{figure}[H] 
	\centering
	\includegraphics[scale=0.35]{../practica_1/images/extremo_a.png} 
	\caption{Extremo a en JFLAP} 
    \label{fig:extremo_a}
\end{figure}

\newpage
\section{Practica 2 - Analizador Léxico}
En esta práctica hemos ideado un lenguaje simple de programación. El lenguaje consta de una sintaxis sencilla que será comprobada con el analizador léxico
que hemos programado. Los elementos de nuestro lenguaje son los siguientes:

\begin{itemize}
	\item \textbf{Función principal: } la función M englobará las instrucciones del programa.
	\item \textbf{Función de lectura por teclado: } la función R abre un flujo de entrada para asignarle un valor a una variable.
	\item \textbf{Función de escritura en pantalla: } la función W escribe por pantalla un valor.
	\item \textbf{Operaciones: } las operaciones permitidas en este lenguaje son la asignación (=), suma (+), resta (-), multiplicación (*), división (/) y módulo (\%).
	\item \textbf{Comentarios: } irán precedidos por una almohadilla (\#).
	\item \textbf{Sentencias: } operaciones, funciones de lectura o escritura terminadas con punto y coma (;).
\end{itemize}

\textcolor{white}{.}

Además, hay una serie de requisitos sintácticos que se han de cumplir:

\begin{itemize}
	\item Todo programa ha de tener una función principal.
	\item No puede haber instrucciones fuera de la función principal.
	\item Todas las sentencias tienen que terminar con un punto y coma (;).
	\item En una asignación se pueden realizar las operaciones oportunas y se pueden utilizar tanto variables como literales.
	\item No se le puede asignar un valor a un literal.
	\item Las funciones R y W deben tener un parámetro.
	\item El único parámetro valido para la función R son las variables.
	\item La función W puede tener cualquier parámetro, incluso una operación.
\end{itemize}

\textcolor{white}{.}

\begin{table}[H]
	\centering
	\begin{tabular}{lllll}
		\multicolumn{1}{c}{\underline{\textbf{Programa correcto}}} & \multicolumn{1}{c}{\underline{\textbf{Programa incorrecto}}} \\
		\multicolumn{1}{c}{\includegraphics[scale=0.5]{../practica_2/images/programa_correcto.png}} & \multicolumn{1}{c}{\includegraphics[scale=0.5]{../practica_2/images/programa_err.png}}
	\end{tabular}
\end{table}

En las fotos anteriores mostramos dos ejemplos que ilustran como sería un programa correcto y otro con errores sintácticos.

A continuación, adjuntamos unas capturas del fichero analizador.lex. La primera imagen corresponde con los flags y contadores que van a ser usados por el analizador para determinar los errores que puedan 
surgir y su correspondiente número de línea para buscar el error y corregir el mismo de una forma más cómoda. 

\begin{figure}[H] 
	\centering
	\includegraphics[scale=0.5]{../practica_2/images/variables.png} 
	\caption{Flags y contadores del analizador léxico} 
    \label{fig:variables}
\end{figure}

La siguiente imagen muestra las expresiones regulares que tendrá en cuenta nuestro analizador. Estas parten de unas básicas a unas más complejas formadas por expresiones anteriores. Se utilizan
para analizar tanto instrucciones sintácticamente correctas como incorrectas. La expresión que más se utiliza es la denominada ``ignore'', cuyo único propósito es que no se tengan en cuenta
los posibles espacios en blanco o tabulaciones entre los distintos elementos de las instrucciones, por ejemplo, los espacios en la siguiente instrucción: a = 2\ \ \ \ + 3;

Las siguientes expresiones son utilizadas para tener en cuenta los posibles errores:
\begin{itemize}
	\item \textbf{asignacion\_err:} se intenta asignar un valor a un literal (4 = a, 3 = 1, etc).
	\item \textbf{lectura\_err:} se intenta asignar un valor por consola a un literal (R(5)).
	\item \textbf{sentencia\_err:} no se escribe punto y coma al final de una sentencia.
\end{itemize}

\begin{figure}[H] 
	\centering
	\includegraphics[scale=0.425]{../practica_2/images/expresiones_regulares.png} 
	\caption{Expresiones regulares del analizador léxico} 
    \label{fig:expresiones_regulares}
\end{figure}

Declaramos las acciones a realizar cuando se encuentran estos elementos que hemos definido anteriormente, es decir, actualizamos el valor de los flags e incrementamos el valor de los contadores para que
quede registrado la línea que se está procesando. Además, mostramos por pantalla el elemento que se ha reconocido en cada momento.

\begin{figure}[H] 
	\centering
	\includegraphics[scale=0.35]{../practica_2/images/acciones.png} 
	\caption{Flags y contadores del analizador léxico} 
    \label{fig:acciones}
\end{figure}

Por último, escribimos las sentencias en C para mostrar la salida del analizador léxico en función de los flags. Si hay algún error sintáctico mostramos el error junto con la línea donde ocurre el mismo.

\begin{figure}[H] 
	\centering
	\includegraphics[scale=0.39]{../practica_2/images/funciones_c.png} 
	\caption{Flags y contadores del analizador léxico} 
    \label{fig:funciones_c}
\end{figure}

A continuación, mostramos las capturas de la salida de nuestro analizador con los ejemplos de programa correcto y programa con errores anteriores.

\begin{table}[H]
	\centering
	\begin{tabular}{lllll}
		\multicolumn{1}{c}{\includegraphics[scale=0.38]{../practica_2/images/main_correcto_ejec.png}} & \multicolumn{1}{c}{\includegraphics[scale=0.38]{../practica_2/images/main_err_ejec.png}}
	\end{tabular}
\end{table}

\newpage

Además, adjuntamos una ejecución para un programa con errores diferentes.

\begin{table}[H]
	\centering
	\begin{tabular}{lllll}
		\multicolumn{1}{c}{\includegraphics[scale=0.5]{../practica_2/images/programa_err_2.png}} & \multicolumn{1}{c}{\includegraphics[scale=0.45]{../practica_2/images/main_err_ejec_2.png}}
	\end{tabular}
\end{table}



\newpage
\section{Practica 3 - Codificador-Decodificador}


\end{document}
