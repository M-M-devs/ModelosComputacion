\input{paquetes.tex}

\title{	
\normalfont \normalsize 
\textsc{\textbf{Modelos de Computación} \\ Grado en Ingeniería Informática \\ Universidad de Granada} \\ [25pt] % Your university, school and/or department name(s)
\horrule{0.5pt} \\[0.4cm] % Thin top horizontal rule
\huge Memoria Prácticas \\ % The assignment title
\horrule{2pt} \\[0.5cm] % Thick bottom horizontal rule
}

% Nombre y apellidos
\author{ 
    Marina Muñoz Cano
    \\
    Mario López González
} 

\date{\normalsize\today} % Incluye la fecha actual

%----------------------------------------------------------------------------------------
% DOCUMENTO
%----------------------------------------------------------------------------------------

\begin{document}

\maketitle % Muestra el Título

\newpage %inserta un salto de página

\tableofcontents % para generar el índice de contenidos
\listoffigures % para generar el índice de imágenes

\newpage

\section{Practica 1 - Relación Ejercicios 1b}

\subsection{Ejercicios Sencillos}

\textbf{a)}  $\{ u \in \{0,1\}^{\ast} $ tales que $\mid u \mid$ $\leq 4 \}$

\begin{figure}[H] 
	\centering
	\includegraphics[scale=0.35]{../practica_1/images/sencillo_a.png} 
	\caption{Sencillo a en JFLAP} 
    \label{fig:sencillo_a}
\end{figure}

Usamos una variable X que puede tomar valor 0, 1 o $\varepsilon$, como son palabras de 0's y 1's de longitud menor o igual que 4, 
el simbolo de partida son cuatro X's. De esta forma se aceptan cadenas de longitud 4 o menos, ya que X puede tomar el valor $\varepsilon$. \\

\textbf{b)}  Palabras con 0's y 1's que no contengan dos 1's consecutivos y que empiecen por un 1 y que terminen por dos 0's.

\begin{figure}[H] 
	\centering
	\includegraphics[scale=0.35]{../practica_1/images/sencillo_b.png} 
	\caption{Sencillo b en JFLAP} 
    \label{fig:sencillo_b}
\end{figure}

La variable A añade 0's tras añadir un 1, para que no contenga dos consecutivos. La variable B va generando el resto de la cadena (Añadiendo 0's o 1)
de forma que la unica forma de acabar la cadena es cuando B vale 0, es decir, con dos 0 consecutivos. 
\\

\textbf{c)}  El conjunto vacío. 

\begin{figure}[H] 
	\centering
	\includegraphics[scale=0.5]{../practica_1/images/sencillo_c.png} 
	\caption{Sencillo c en JFLAP} 
    \label{fig:sencillo_c}
\end{figure}

No mostramos ejemplos, ya que JFLAP no nos lo permite, al no detectar ningun simbolo como terminal.

\begin{figure}[H] 
	\centering
	\includegraphics[scale=0.5]{../practica_1/images/sencillo_c_err.png} 
	\caption{Sencillo c error en JFLAP} 
    \label{fig:sencillo_c_err}
\end{figure}

\textbf{d)}  El lenguaje formado por los números naturales.

El lenguaje genera las cifras del numero final de izquierda a derecha, no puediendo empezar por 0. De esta forma generamos un número que comprende
entre el 1 y el 9. Tras esto se le puede agregar cualquier número entre 0 y 9.

\begin{figure}[H] 
	\centering
	\includegraphics[scale=0.35]{../practica_1/images/sencillo_d.png} 
	\caption{Sencillo d en JFLAP} 
    \label{fig:sencillo_d}
\end{figure}

\textbf{f)}  $\{ a^{n} b^{2n} c^{m} \in \{a,b,c\}^{\ast}$ con $n,m > 0\}$

Con la variable A, por cada a generamos dos b's. Tras generar las n a's y las 2n b's generamos m c's con la variable C.

\begin{figure}[H] 
	\centering
	\includegraphics[scale=0.35]{../practica_1/images/sencillo_f.png} 
	\caption{Sencillo f en JFLAP} 
    \label{fig:sencillo_f}
\end{figure}

\textbf{g)} $\{ a^{n} b^{m} a^{n} \in \{a,b\}^{\ast}$ con $m,n > 0\}$

Con el símbolo de partida S, generamos el número de a's mínimo que admite el lenguaje. Tras esto se puede seguir añadiendo a's a ambos lados o incluir las m b's,
siendo obligatorio añadir como mínimo una b. 

\begin{figure}[H] 
	\centering
	\includegraphics[scale=0.4]{../practica_1/images/sencillo_g.png} 
	\caption{Sencillo g en JFLAP} 
    \label{fig:sencillo_g}
\end{figure}

\textbf{h)}  Palabras con 0's y 1's que contengan la subcadena 00 y 11.

\begin{figure}[H] 
	\centering
	\includegraphics[scale=0.35]{../practica_1/images/sencillo_h.png} 
	\caption{Sencillo h en JFLAP} 
    \label{fig:sencillo_h}
\end{figure}

Con las variables S generamos las dos subcadenas necesarias, representadas por X e Y. Con las variables anteriores se añaden variables Z a ambos lados de las
cadenas, siendo posible sustituilas por 0's, 1's o $\varepsilon$. De esta forma, se generan cadenas de 0's y 1's que contienen en cualquier orden las subcadenas 00 y 11. \\

\textbf{i)}  Palíndromos formados con las letras a y b.

Para formar los palíndromos con el símbolo de partida generamos dos a's o dos b's. En las cadenas generadas incluimos la variable S para que se puedan generar las cadenas de 
a's y b's necesarias para crear el palíndromo. Además, incluimos las reglas de produccion que sustituyen S por a y b para poder generar palíndromos impares y otra regla,
S $\rightarrow$ $\varepsilon$, para generar pares.

\begin{figure}[H] 
	\centering
	\includegraphics[scale=0.375]{../practica_1/images/sencillo_i.png} 
	\caption{Sencillo i en JFLAP} 
    \label{fig:sencillo_i}
\end{figure}

\newpage

\subsection{Ejercicios Intermedios}

\textbf{a)}  $\{ uv \in \{0,1\}^{\ast} $ tales que $u^{-1}$ es un prefijo de $v\}$

Partimos de las variables XY, con X generaremos u y $u^{-1}$, y con Y generaremos el resto de v. El palíndromo par, lo generamos de forma
similar al que generamos en el ejercicio sencillo i con la excepción de no formar palíndromos impares.
v se genera a partir de la variable Y con 0's y con 1's sin importar el orden.

\begin{figure}[H] 
	\centering
	\includegraphics[scale=0.35]{../practica_1/images/intermedio_a.png} 
	\caption{Intermedio a en JFLAP} 
    \label{fig:intermedio_a}
\end{figure}

\textbf{b)}  $\{ ucv \in \{a,b,c\}^{\ast} $ tales que $u$ y $v$ tienen la misma longitud$\}$

\begin{figure}[H] 
	\centering
	\includegraphics[scale=0.35]{../practica_1/images/intermedio_b.png} 
	\caption{Intermedio b en JFLAP} 
    \label{fig:intermedio_b}
\end{figure}

\textbf{c)}  $\{ u1^{n} \in \{0,1\}^{\ast} $ donde $\mid u \mid$ $= n \}$
 
\begin{figure}[H] 
	\centering
	\includegraphics[scale=0.35]{../practica_1/images/intermedio_c.png} 
	\caption{Intermedio c en JFLAP} 
    \label{fig:intermedio_c}
\end{figure}

\textbf{d)}  $\{ a^{n} b^{n} a^{n+1} \in \{a,b\}^{\ast}$ con $n > 0\}$

La variable X se mueve hacia la derecha de la ultima b para generar una a y una b, una vez hecho esto, se intercambia por una Y, que se desplaza hasta la izquierda de la primera b para generar una a y se intercambia por una X. Esto se repite hasta que generamos la cadena al completo.

\begin{figure}[H] 
	\centering
	\includegraphics[scale=0.35]{../practica_1/images/intermedio_d.png} 
	\caption{Intermedio d en JFLAP} 
    \label{fig:intermedio_d}
\end{figure}

\subsection{Ejercicios Dificiles}

\textbf{a)}  $\{ u0v \in \{0,1\}^{\ast} $ tales que $u^{-1}$ es un prefijo de $v\}$

Partimos de dos variables X e Y. Con la variable X, vamos generando $u$ y $u^{-1}$, una vez generadas, entre $u$ y $u^{-1}$ añadimos un 0. Con la variable Y generamos el resto de $v$.


\begin{figure}[H] 
	\centering
	\includegraphics[scale=0.35]{../practica_1/images/dificil_a.png} 
	\caption{Dificil a en JFLAP} 
    \label{fig:dificil_a}
\end{figure}

\subsection{Ejercicios Extremos}

\textbf{a)}  $\{ ww$ con $w \in \{0,1\}^{\ast}\}$

Usamos tres variables:
\begin{itemize}
\item \textbf{M}: para ir insertando en orden la cadena w. A la izquierda de M va quedando en orden, mientras que, a su derecha va quedando invertida.
\item \textbf{I}: para ir invirtiendo la parte la izquierda de M. Se va desplazando hacia la izquierda intercambiando dos simbolos terminales entre si.
\item \textbf{F}: marca el final de la cadena de la izquierda para saber cuando I tiene que parar de invertir. 
\end{itemize}
De esta forma, cada vez que se inserta un 1 o 0 con M, la funcion de I es ir desplazándolo dejándolo al comienzo de la cadena (marcado por F).

\begin{figure}[H] 
	\centering
	\includegraphics[scale=0.35]{../practica_1/images/extremo_a.png} 
	\caption{Extremo a en JFLAP} 
    \label{fig:extremo_a}
\end{figure}

\section{Practica 2 - Analizador Léxico}

\section{Practica 3 - Codificador-Decodificador}


\end{document}
